\documentclass[../../main.tex]{subfiles}
\begin{document}
Binnen UML worden volgende diagrammen gebruikt bij het \imp{dynamisch modelleren}:
\begin{itemize}

	\item \textbf{Use Case Diagrammen} \\ Deze beschrijven de manier waarop een gebruiker het systeem wenst te gebruiken. Het beschrijft dus problemen waar de gebruiker een oplossing voor wenst.
	\begin{itemize}
		\item Beschrijft het probleem: functionele vereisten
		\item Kan niet gebruikt worden voor model van oplossing \\ (omvat immers geen toekomstige use cases)
	\end{itemize}
	\item \textbf{Sequentie Diagram} \\ Beschrijft hoe objecten in het systeem samenwerken.
	\begin{itemize}
		\item Beschrijf de oplossing
	\end{itemize}
	\item \textbf{Toestandsdiagrammen} \\ Deze worden gebruikt om de \imp{toestanden} waarin een object zich kan bevinden te beschrijven. Het legt ook vast wanneer welke overgangen mogelijk zijn.
	\begin{itemize}
		\item Beschrijven probleemdomein
		\item Beschrijven oplossing
	\end{itemize}
	\item \textbf{Activiteitsdiagrammen} \\ Procesgerichte beschrijvingen om het gedrag van een systeem, bestaande uit meerdere objecten, te modelleren.
	\begin{itemize}
		\item Workflow modelleren
		\item Concretiseren use case
	\end{itemize}
	\item \textbf{Interactiediagrammen} \\ Deze worden gebruikt voor het modelleren van de interacties tussen de objecten.
	\begin{itemize}
		\item Gebonden aan OOP
		\item Beschrijven oplossingsdomein
		\item Om ze toch te gebruiken in probleemdomein: \imp{Object-Event tabel}
	\end{itemize}
\end{itemize}


\subsection{Positioneren technieken}
\begin{tabular}{|l|c|c|c|}
\hline
 & \multicolumn{2}{c|}{Problem Domain} & Solution Domain \\
\hline
 & Domain Knowledge (*) & IS FUnctionality (**) & (***) \\
\hline
Class Diagram & Y & Y & Y \\
\hline
Use Case Diagram & • & Y & Y \\
\hline
Sequence Diagram & • & • & Y \\
\hline
Finite State Diagram & Y & • & Y \\
\hline
Activity Diagram (BPMN) & Y & • & Y \\
\hline
\end{tabular}
\begin{description}
	\item \textbf{*}: Rij 2 van raamwerk Zachman
	\item \textbf{**}: Rij 3 van raamwerk Zachman
	\item \textbf{***}: Rij 4,5,6 van raamwerk Zachman
\end{description}
\subsection{Diagramtechnieken in deze cursus}
\begin{itemize}
	\item \textbf{Domeinmodel}
	\begin{itemize}
		\item Klassendiagrammen
		\item Toestandsdiagrammen
	\end{itemize}
	\item \textbf{Functionaliteit}
	\begin{itemize}
		\item Use cases
	\end{itemize}
	\item \textbf{Ontwerp}
	\begin{itemize}
		\item Sequentie- en interactiediagrammen
	\end{itemize}
\end{itemize}
\end{document}