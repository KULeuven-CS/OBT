\documentclass[../../main.tex]{subfiles}
\begin{document}
Zie ook hoofdstuk 8 in het handboek\\
\\
Use cases bieden een \imp{gebruikersperspectief} op het systeem. Ze worden gebruikt voor \imp{requirements elicitation}: de techniek tracht de functionele vereisten van het systeem te achterhalen.
\begin{itemize}
	\item Snijdt doorheen alle architectuurlagen heen
	\item Geen goede basis voor systeemontwerp
\end{itemize}
\subsection{Controleren}
Bij het opstellen van een use case kunnen we volgende criteria gebruiken om de volledigheid van de analyse te controleren:
\begin{enumerate}
	\item Elke actor heeft minstens \'e\'en use case om met het systeem te interageren.
	\item Elke bedrijfsklasse heeft een use case waarmee men de lijst van objecten van die klasse kan opvragen
	\item Elke bedrijfsklasse heeft een use case waarmee men de details van een gegeven object uit de klasse kan opvragen
	\item Elke bedrijfsgebeurtenis heeft een use case waarmee men de uitvoering kan triggeren
\end{enumerate}
\end{document}