\documentclass[../main.tex]{subfiles}
\begin{document}
Het neerschrijven van het model of het lastenboek kan op verschillende manieren.
\begin{itemize}
	\item \textbf{Natuurlijke taal:} Vaak enkel voor de eerste inventaris van de wensen van de gebruiker. Eenvoudig maar niet effici\"ent.
	\item \textbf{UML:} Industriestandaard voor het beschrijven van een informatiesysteem. In eerste plaats bedoeld voor het beschrijven van de oplossing, maar kunnen ook het probleemdomein beschrijven.
\end{itemize}
Om de gebruiker ook in staat te stellen het model te begrijpen, beperken we ons tot een subset van de beschikbare technieken. We laten wat bedoeld is voor de technische implementatie.\\
\\
We zullen de vertaling maken naar \imp{algebra}. We gebruiken het wiskundig perspectief om
\begin{itemize}
	\item \textbf{Betekenis} van het schema te defini\"eren
	\item \textbf{Abstract denken} te stimuleren
	\item Modellen te \textbf{valideren}
\end{itemize}

\subsection{Terminologie}
\begin{description}
	\item[Modelleringstaal] Geheel van technieken bedoeld voor het neerschrijven van (software-) specificaties. \\
	(Bijvoorbeeld \textbf{UML})
	\item[Modelleringstechniek] Samenhangend geheel van symbolen en regels die toelaten een bepaald aspect van de software te modelleren. \\
	(Bijvoorbeeld \imp{UML Class Diagram}, \imp{ER Modelling}, \imp{Data Flow Diagram})
	\item[Diagramma, schema of model] Model of tekening opgesteld volgens een bepaalde voorafbepaalde  techniek. \\
	(Bijvoorbeeld \imp{UML klassendiagram})
\end{description}
\subsection{Klassen en objecten}
\begin{itemize}
	\item \textbf{Wiskundige voorstelling:} We noteren de verzameling als een \imp{ovaal}. De elementen in de verzameling vormen \imp{punten} in deze ovaal.
	\item \textbf{UML:} Verzamelingen worden voorgesteld als \imp{rechthoeken} met daarin de naam van de verzameling. We noemen de verzameling een \imp{klasse} en de elementen \imp{objecten}.
\end{itemize}

\subsection{Attributen}
Bij het defini\"eren van een verzameling, beschrijven we de elementen van die verzameling aan de hand van hun eigenschappen. Deze worden de \imp{attributen} van het object genoemd.
\begin{itemize}
	\item Eigenschappen die object altijd bezit
	\item Attribuutnaam ligt vast
	\item Waarde kan eventueel veranderen
\end{itemize}
\section{Associaties}
\subsection{Principe}
Elementen van een verzameling kunnen een \imp{verband} hebben met andere elementen uit een andere verzameling.
\subsubsection{Wiskundige benadering}
We kunnen de verbanden defini\"eren als afbeeldingen tussen verzamelingen.
\begin{description}
	\item[Afbeelding]  Een afbeelding $f : A \rightarrow B$ is een functie die een element uit de verzameling van $A$ afbeeldt op elementen van verzameling $B$. $A$ is het \imp{domein} van de afbeelding $f$ en $B$ is het \imp{beeld} van de afbeelding $f$.
\end{description}

\begin{equation*}
f: A \rightarrow B:a \mapsto f(a) \textrm{met } a \in A \textrm{ en } f(a) \subseteq B
\end{equation*}
Het verband kan ook in de omgekeerde richting worden beschouwd. Vaak heeft het verband in elke richting een eigen naam.
\subsubsection{UML benadering}
We noemen het verband tussen twee of meer klassen een \imp{associatie}. Deze wordt weergegeven als een \imp{lijn} met de naam van het verband naast de lijn geschreven. Wanneer het verband in beide richting een naam krijgt, noemt men dit de \imp{rollen} van de associatie.
\\\\
\begin{ex}{rollen}
Veronderstel twee klassen: werknemer en departement. \\
$\rightarrow$ : Werknemer \textit{werkt in} departement \\
$\leftarrow$ : Departement \textit{heeft} werknemer
\end{ex}
\subsection{Karakter van een verband}
\subsubsection{Verplicht/optioneel karakter (Wiskunde)}
Een verband kan ofwel verplicht of optioneel zijn.
\begin{itemize}
	\item \textbf{Verplicht:} Een afbeelding is verplicht indien elk element van $A$ verplicht een beeld in $B$ heeft. \\
	\begin{equation*}
	\boxed{f:A \rightarrow B \textrm{ is verplicht} \Leftrightarrow \forall a \in A:f(a) \neq \emptyset}
	\end{equation*}
	\item \textbf{Optioneel:} Een afbeelding is optioneel indien niet elk element van $A$ verplicht een element in $B$ heeft.
\end{itemize}
Het \imp{verplicht} of \imp{optioneel} karakter wordt gedefinieerd voor elk object op elk moment. Deze verandert \textbf{niet} doorheen de tijd. De interpretatie die hier gegeven wordt aan een verplicht verband wordt in de temporele logica gemodelleerd met de $\forall$-kwantor.
\\
\\
UML maakt geen gebruik van temporele kwantoren. Een verplicht verband moet \textbf{altijd} geldig zijn. Uitspraken met de $\exists$-kwantor zijn niet mogelijk.
\subsubsection{kardinaliteit van een verband (UML)}
\begin{description}
	\item[kardinaliteit] De kardinaliteit van een verband $f$ zegt wat het maximum aantal elementen is waarop een element uit het domein wordt afgebeeld door $f$.
\end{description}

\begin{itemize}
	\item \textbf{kardinaliteit maximum 1} \\
	Een verband $f: A \rightarrow B$ heeft een \imp{carinaliteit van maximum 1} indien elke $a \in A$ op maximum \'e\'en $b \in B$ wordt afbeeld.\\
	\begin{equation*}
	\boxed{f:A \rightarrow B \textrm{ heeft een kardinaliteit van maximum 1} \Leftrightarrow \forall a \in A:|f(a)| \leq 1}
	\end{equation*}
	\item \textbf{kardinaliteit many} \\
	Een verband $f: A \rightarrow B$ heeft een \imp{carinaliteit van many} indien elke $a \in A$ mogelijks op meerdere elementen $b_i \in B$ wordt afbeeld.\\
	\begin{equation*}
	\boxed{f:A \rightarrow B \textrm{ heeft een kardinaliteit van many} \Leftrightarrow \forall a \in A:|f(a)| \in \mathbb{N}}
	\end{equation*}
\end{itemize}

Het verplicht of optioneel karakter wordt dan als volgt weergegeven:
\begin{center}
\begin{tabular}{|l|l|l|}
\hline
$f:A \rightarrow B$ & & Schrijf langs $f$, aan de kant van $B$ \\ \hline
$f$ is optioneel & $f$ heeft carinaliteit 1 & [0..1] \\
& $|f(a)| \leq 1$ & \\ \hline
$f$ is optioneel & $f$ heeft carinaliteit many & [0..*] (afgekort *) \\ \hline \hline
$f$ is verplicht & $f$ heeft carinaliteit 1 & [1..1] (afgekort 1)\\
$|f(a) > 0|$ & $|f(a)| \leq 1$ &  $|f(a)|=1$ \\ \hline
$f$ is verplicht & f heeft carinaliteit many & [1..*]  \\ \hline
\end{tabular}
\end{center}

\subsection{Navigeren over meerdere associaties}
Waneer het beeld van het ene verband $f$ gelijk is aan het domein van een ander verband $g$ kunnen we verbanden \imp{combineren} tot \'e\'en verband $g \bullet f$
\begin{center}
$f:A \rightarrow B$ en $g:B \rightarrow C$ dan geldt $g \bullet f: A \rightarrow C: a \mapsto g(f(a))$
\end{center}
Het optioneel of verplicht karakter en de kardinaliteit van het \imp{samengesteld verband} kunnen afgeleid worden uit de eigenschappen van de deelverbanden.\\\\
\noindent
\begin{ex}{samengesteld verband}
Veronderstel dat $f$ kardinaliteit 1 heeft en $g$ kardinaliteit many. \\
Dan geldt $\forall a \in A$ dat er maximum \'e\'en element zit in $f(a)$, waarbij $f(a) \subseteq B$. Voor elk element $b \in B$ en dus ook $b \in f(a)$ geldt dat er mogelijks meerdere elementen zijn in $g(b)$. We kunnen besluiten dat er in $g(f(a)) = (g \bullet f)(a)$ ook mogelijks meerdere elementen zitten.
\end{ex}
\end{document}