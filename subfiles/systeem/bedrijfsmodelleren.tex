\documentclass[../../main.tex]{subfiles}
\begin{document}

Het vinden van de juiste klassen en verbanden is een van de moeilijkste elementen van modelleren. Verschillende technieken maken het ons gemakkelijker.
\subsection{Lagen}
Klassen worden georganiseerd in \imp{lagen}. Binnen dezelfde laag kan er gebruik gemaakt worden van diensten van onderliggende lagen. Wijzigingen propageren zich dus enkel opwaarts. De onderste lagen zijn dan ook de meest stabiele.\\
\\
Typische lagen:
\begin{itemize}
	\item \textbf{User Interface} \\
	\\
	\begin{ex}{gebruikersinterface}
	Knoppen, velden
	\end{ex}
	\item \textbf{Applicatielogica}\\
	\\
	\begin{ex}{applicatielogica}
	Berekenen van de datum
	\end{ex}
	\item \textbf{Bedrijfsobjecten}
	\begin{itemize}
		\item Logica\\
		\\
		\begin{ex}{logica}
		Regels in verband met het verlengen van de uitleentermijn
		\end{ex}
		\item Data\\
		\\
		\begin{ex}{data}
		Titel van een boek
		\end{ex}
	\end{itemize}
\end{itemize}
We zien dat de lagen de volgorde van variabiliteit volgen.

\subsection{Soorten specificaties: Zachman}
Voor een informatiesysteem kan men verschillende plannen maken. Het \imp{Zachman framework} is een classificatieschema voor complexe informatiesystemen.
\begin{itemize}
	\item Laat toe onderdelen te bestuderen zonder context uit het oog te verliezen
	\item Biedt overzicht van complex systeem
	\item Combineert twee perspectieven:
	\begin{itemize}
		\item Organisatorisch perspectief
		\item Informatiesysteem perspectief
	\end{itemize}
	Laat toe de bedrijfs- en informatiestrategie op elkaar af te stemmen
\end{itemize}
Voor de juiste indeling, zie schema in de cursus op pagina 51.\\
\subsubsection{Rij 2: Het enterprise-model perfectief}
Dit perspectief stemt overeen met dat van de \textbf{eigenaar} of \textbf{opdrachtgever}. Dit model geeft een overzicht van:
\begin{itemize}
	\item Werking van de organisatie
	\item Businessobjecten
	\item Geldende regelgeving
\end{itemize}
Omdat dit deel zich bezig houdt met het beschrijven van de \textit{business}, wordt dit ook vaak \imp{business modelling} genoemd.\\
\\
\begin{ex}{kul}
Het onderwijs- en examenreglement vormt een onvolledige en ongestructureerde beschrijving van de \textit{business} van de KUL.
\end{ex}
\subsubsection{Rij 3: Informatiesysteem-model perspectief}
Dit perspectief stemt overeen met dat van de \textbf{ontwerper}. Deze heeft kennis van zowel de \textbf{bedrijfsaspecten} als de \textbf{technische aspecten}. \\
\\
Hier wordt de gewenste functionaliteit bepaalt, rekening houdend met de beperkingen van de omgeving waarin het systeem operationeel is.

\subsection{Doel van bedrijfsmodellering}
Tot dusver bekeken we het perspectief van de opdrachtgever: het lastenboek. Dit omvat de bovenste drie lagen van het framework.
\begin{itemize}
	\item Bepalen scope
	\item Vastleggen bedrijfsregels
	\item Bepalen van de informatiesysteemondersteuning
\end{itemize}
De oefeningen besloegen slechts het enterprise model (rij 2).

\subsection{Het bedrijfsmodel}
Het bedrijfsmodel...
\begin{itemize}
	\item \textbf{Als kernmodel} \\
	Het bedrijfsmodel definieert de bedrijfsobjecten en legt tevens de bedrijfslogica vast. Dit vormt de kern van het informatiesysteem.
	\item \textbf{Als middel tot alignering onder de gebruikers} \\
	Het is een gestructureerd en meer formele voorstelling van alle bedrijfsconcepten, -regels en -processen. Het vormt een gemeenschappelijke taal onder de betrokkenen.
	\item \textbf{Als middel tot een robuust systeem} \\
	Het bedrijfsmodel beschrijft naast het probleemdomein ook de kern van de oplossing. Een beter begrip leidt tot betere software. (zie fabel bootverhuur)
\end{itemize}
\subsection{Het enterprise model}
Zodra duidelijk is welke specificaties betrekking hebben met het bedrijfsmodel, kunnen we de klassendefinities opstellen. \\
\\
Dit proces is normaal al meermaals aan bod is gekomen. Wie het toch nog wil herhalen kan het nalezen op pagina 56 van de cursus.\\
\\
Idealiter beschrijft het bedrijfsmodel ook de dynamische aspecten. Hiervoor zijn twee invalshoeken:
\begin{itemize}
	\item Gedragsaspecten per business beschrijven door middel van een  \imp{FSM}.
	\item Opstellen van bedrijfsmodellen: dit is een keten van activiteiten die wordt ontplooid met het oog op het realiseren van de bedrijfsdoelstellingen.
\end{itemize}
Beide processen zijn complementair aan elkaar.
\end{document}