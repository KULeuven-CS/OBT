\documentclass[../../main.tex]{subfiles}
\begin{document}

Door het gebruik van meerdere technieken voor het beschrijven van \'e\'en probleem, komen sommige elementen meermaals voor.\\
\\
\begin{ex}{ }
De klassen die aan bod komen in de FSM zijn ook gedefinieerd in het klassendiagram. De events die overgangen bepalen zijn ook onderdeel van de OET.
\end{ex}
\\
Het spreekt voor zich dat we moeten controleren of deze wel consistent zijn. Dit gebeurt aan de hand van kwaliteitscontrole.

\subsection{Randvoorwaarden}
De taal het domein en het publiek vormen de \imp{randvoorwaarden} die de kwaliteit van het model be\"invloeden.
\begin{itemize}
	\item \textbf{Relatie taal-domein}: Is de taal geschikt voor het neerschrijven van het model?
	\item \textbf{Relatie taal-publiek}: Is de taal geschikt om te communiceren met het publiek?
	\item \textbf{Relatie publiek-domein}: Het juiste publiek moet gekozen worden om daarop een model te baseren.
\end{itemize}
\subsection{Types van kwaliteitscontrole}
Gegeven een domein, taal en publiek. We kunnen dan volgende types van kwaliteitscontrole onderscheiden:
\begin{itemize}
	\item \textbf{Syntax:} voldoet het madel aan de vormeisten opgelegd door de taal?
	\item \textbf{Semantiek:} klopt de betekenis van het model?
	\begin{itemize}
		\item \textbf{Validiteit:} Klopt het?
		\item \textbf{Volledigheid:} Is alles opgenomen?
	\end{itemize}
	\item \textbf{Pragmatiek:} kan het model begrepen worden door het publiek?
\end{itemize}
\subsection{Middelen tot kwaliteitscontrole}
De meeste modelleringstools bieden een aantal hulpmiddelen voor de controle op kwaliteit:
\subsubsection{Syntax}
Meestal laat het gebruikte programma niet toe syntaxfouten te maken.
\subsubsection{Semantiek}
Voor semantische controle is het belangrijk dat de betekenis van de concepten in de modelleringstaal eenduidig is gemodelleerd. Het is van groot belang dat de uitspraken elkaar niet tegenspreken.
\begin{itemize}
	\item \textbf{Externe validiteit:} Controle op consistentie met het domein
	\item \textbf{Interne validiteit:} Controle op consistentie met de reeds expliciete of afgeleide uitspraken
\end{itemize}
Er bestaan drie manieren waarop aan consistentiecontrole wordt gedaan:
\begin{itemize}
	\item Consistentie via \textbf{analyse}: Controleren aan de hand van een algoritme.
	\item Consistentie via \textbf{monitoring}: Bij het toevoegen direct controleren of er geen tegenspraak is ontstaan.
	\item Consistentie via \textbf{constructie} Bij het toeveogen direct afgeleide uitspraken vormen en toevoegen.
\end{itemize}
Dergelijke controle is enkel mogelijk als er voldoende duidelijke regels bestaan.
\subsubsection{Pragmatiek}
Voor de controle op pragmatiek bieden modelleringsprogramma's eveneens verschillende hulpmiddelen.
\end{document}