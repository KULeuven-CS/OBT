\documentclass[../../main.tex]{subfiles}
\begin{document}
Een oplossing die we mits aanpassing kunnen toepassing in verschillende context wordt een \imp{pattern} genoemd.\\
\\
Het omzetten van een verband naar een tussenliggende klasse is een bewerking die herhaaldelijk wordt toegepast. Formeel zijn er verschillende definities, waarvan de meer concrete luidt:
\begin{quote}
	A pattern is the abstraction from a concrete form which keeps recurring in specific non-arbitrary contexts.
\end{quote}
Patterns zijn interessant omdat ze domeinkennis bevatten en de grondgedachte documenteren:
\begin{itemize}
	\item Hergebruiken wijsheid en ervaring
	\item Bieden naast oplossing ook context
	\item Ontleden de oplossing
\end{itemize}
\subsection{Soorten patterns}
Patterns kunnen worden opgedeeld volgens de fasen van de systeemontwikkelign:
\begin{itemize}
	\item \textbf{Analyse:} Analyse-patterns \\
	Pattern waarvan de vorm beschreven wordt met behulp van voorwaarden en concepten uit het applicatiedomein.
	\item \textbf{Design:} Design-patterns \\
	Pattern waarvan de vorm beschreven wordt software design constructies zoals objecten, klassen, overerving, ...
	\item \textbf{Implementatie:} Code-patterns \\
	Patters waarvan de vorm beschreven wordt met programmeertaalinstructies.
\end{itemize}
\noindent
Patterns kunnen toegepast worden op verschillende domeinen.

\subsection{Elementen van een pattern}
Een pattern kan op verschillende manieren worden beschreven. Twee veelvoorkomende vormen zijn:
\begin{itemize}
	\item \textbf{Alexandrian Form}: doorlopende tekst
	\item \textbf{Complien Form}: gestructureerd
\end{itemize}
De \imp{Complien Form} bestaat uit volgende elementen:
\begin{itemize}
	\item \textbf{Naam} referentie naar het pattern voor het gemakkelijk terugvinden.
	\item \textbf{Probleem} Uiteenzetting van het probleem
	\item \textbf{Krachten} Welke krachten en beperkingen spelen
	\item \textbf{Oplossing} Statische relaties en dynamische regels beschrijven hoe de gewenste Uitkomst kan worden gerealiseerd
	\item \textbf{Voorbeelden} Ter illustratie
	\item \textbf{Resulteren contex} Configuratie van het systeem nadat het pattern is toegepast
	\item \textbf{Rationale} Verklaring van de stappen of regels en verklaring waarom het werkt.
	\item \textbf{Gerelateerde patterns} Andere patterns delen vaak dezelfde krachten.
	\item \textbf{Gekende gebruiken} Voorkomen van het pattern binnen bestaande systemen
\end{itemize}
\subsection{Anti-patterns}
Een pattern beschrijft een \textit{best practice}, maar met een \imp{anti-pattern} kan het tegengestelde bieden. Er zijn twee soorten:
\begin{itemize}
	\item Slechte oplossing beschrijven en de problemen die daarmee verbonden zijn
	\item Slechte situatie voorkomen en hoe vandaar gewerkt kan worden naar een goede oplossing
\end{itemize}
Wanneer gerelateerde patterns aan elkaar worden gewoven vormt men een \imp{pattern language}. Deze vormen samen de oplossing voor een complex probleem.
\subsection{Pattern mining en PLoP}
Het proces van zoeken naar patterns om ze te documenteren wordt \imp{pattern-mining} genoemd.\\
\\
\imp{PLoP} of \imp{Pattern Languages of Programming}-conferenties zijn een mogelijkheid voor de auteur om hun patterns te laten beoordelen door collega's.
\end{document}