\documentclass[../main.tex]{subfiles}
\begin{document}
De hoofdstukken behandeld door professor Snoeck beslagen de analysefase binnen het ontwerp van bedrijfstoepassingen. Deze fase bevindt zich onder volgende fase die overlopen worden bij dergelijke ontwikkeling.

\section{Ontwikkelingsfasen}
\begin{enumerate}
	\item \textbf{Definitiefase}
	\begin{itemize}
			\item Project wordt afgebakend
			\item Er wordt een \imp{business-case} opgesteld (kosten/baten analyse)
			\item Verschillende alternatieven worden bekeken
	\end{itemize}
	\item \textbf{Analysefase}
	\begin{itemize}
		\item Vereisten worden in kaart gebracht
		\item Onderscheid tussen \imp{high-level analyse} en \imp{gedetailleerde analyse}
		\item Onderscheid \imp{requirements gathering} en \imp{requirements modelling}
	\end{itemize}
	\item \textbf{Ontwerpfase}
	\begin{itemize}
		\item Er wordt ontwerp gemaakt van het systeem
		\item Gebaseerd op de gedetailleerde analyse
	\end{itemize}
	\item \textbf{Bouwfase}
	\begin{itemize}
		\item Programmeren en testen van het systeem
	\end{itemize}
	\item \textbf{Testfase}
	\begin{itemize}
		\item Formele testfase
		\item Acceptatietest uitgevoerd door gebruikers
	\end{itemize}
	\item \textbf{Gebruiksfase}
	\begin{itemize}
		\item Wordt ook \imp{productiefase} genoemd
		\item Zodra het systeem formeel aanvaard is door opdrachtgever
	\end{itemize}
	\item \textbf{Onderhoudsfase}
	\begin{itemize}
		\item Zodra het systeem in gebruik genomen is
		\item Verder op punt stellen op basis van problemen/vragen van gebruikers
	\end{itemize}

\end{enumerate}
\noindent
De manier waarop men bovenstaande fasen doorloopt is afhankelijk van de gekozen \imp{systeemontwikkelingsmethode}. Enkele populaire methodes zijn \imp{Waterfall}, \imp{eXtreme programming}, \imp{Agile}, \imp{RUP}.

\section{Analysefase}
De focus van dit deel van de cursus ligt op de \imp{Analysefase}. Binnen deze fase zijn twee onderdelen te onderscheiden: \imp{verzamelen van requirements} en \imp{modelleren van requirements}.

\begin{description}
	\item[Requirements gathering] De businessanalist probeert zich een inzicht te vormen van de dominante activiteiten. Pas wanneer deze hierover beschikt kan hij de vereisten formeel modelleren.
	\item[Requirements modelling] De analist kan nu de functionele vereisten op een gedetailleerde manier vastleggen. Hierdoor kunnen inconsistenties en onvolledigheden worden opgemerkt.
\end{description}
\noindent
Hoofdstukken 2 en 3 in deze samenvatting (respectievelijk 3 en 4 in de cursus) gaan over het modelleren. In hoofdstuk 4 (5 in de cursus) wordt er vanuit een breder perspectief naar de belangrijke aspecten van het analyseproces in zijn geheel gekeken.


\end{document}